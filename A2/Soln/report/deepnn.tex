%------------------------------------------------------------------------------------
%	PACKAGES AND OTHER DOCUMENT CONFIGURATIONS
%------------------------------------------------------------------------------------

\documentclass{article}

\usepackage{fancyhdr} % Required for custom headers
\usepackage{lastpage} % Required to determine the last page for the footer
\usepackage{extramarks} % Required for headers and footers
\usepackage[usenames,dvipsnames]{color} % Required for custom colors
\usepackage{graphicx} % Required to insert images
\usepackage{subcaption}
\usepackage{listings} % Required for insertion of code
\usepackage{courier} % Required for the courier font
% Optional Packages
\usepackage{amsmath}
\usepackage{amssymb}
\usepackage{float}
\usepackage{algorithm}
\usepackage[noend]{algpseudocode}


% Margins
\topmargin=-0.45in
\evensidemargin=0in
\oddsidemargin=0in
\textwidth=6.5in
\textheight=9.0in
\headsep=0.25in

\linespread{1.1} % Line spacing

% Set up the header and footer
\pagestyle{fancy}
\lhead{\hmwkAuthorName} % Top left header
\chead{\hmwkClass\ : \hmwkTitle} % Top center head
%\rhead{\firstxmark} % Top right header
\lfoot{\lastxmark} % Bottom left footer
\cfoot{} % Bottom center footer
\rfoot{Page\ \thepage\ of\ \protect\pageref{LastPage}} % Bottom right footer
\renewcommand\headrulewidth{0.4pt} % Size of the header rule
\renewcommand\footrulewidth{0.4pt} % Size of the footer rule

\setlength\parindent{0pt} % Removes all indentation from paragraphs


%------------------------------------------------------------------------------------
%	DOCUMENT STRUCTURE COMMANDS
%	Skip this unless you know what you're doing
%------------------------------------------------------------------------------------

% Header and footer for when a page split occurs within a problem environment
\newcommand{\enterProblemHeader}[1]{
	%\nobreak\extramarks{#1}{#1 continued on next page\ldots}\nobreak
	%\nobreak\extramarks{#1 (continued)}{#1 continued on next page\ldots}\nobreak
}

% Header and footer for when a page split occurs between problem environments
\newcommand{\exitProblemHeader}[1]{
	%\nobreak\extramarks{#1 (continued)}{#1 continued on next page\ldots}\nobreak
	%\nobreak\extramarks{#1}{}\nobreak
}

\setcounter{secnumdepth}{0} % Removes default section numbers
\newcounter{homeworkProblemCounter} % Creates a counter to keep track of the number of problems
\setcounter{homeworkProblemCounter}{0}

\newcommand{\homeworkProblemName}{}
\newenvironment{homeworkProblem}[1][Problem \arabic{homeworkProblemCounter}]{ % Makes a new environment called homeworkProblem which takes 1 argument (custom name) but the default is "Problem #"
	\stepcounter{homeworkProblemCounter} % Increase counter for number of problems
	\renewcommand{\homeworkProblemName}{#1} % Assign \homeworkProblemName the name of the problem
	\section{\homeworkProblemName} % Make a section in the document with the custom problem count
	\enterProblemHeader{\homeworkProblemName} % Header and footer within the environment
}{
	\exitProblemHeader{\homeworkProblemName} % Header and footer after the environment
}

\newcommand{\problemAnswer}[1]{ % Defines the problem answer command with the content as the only argument
	\noindent\framebox[\columnwidth][c]{\begin{minipage}{0.98\columnwidth}#1\end{minipage}} % Makes the box around the problem answer and puts the content inside
}

\newcommand{\homeworkSectionName}{}
\newenvironment{homeworkSection}[1]{ % New environment for sections within homework problems, takes 1 argument - the name of the section
	\renewcommand{\homeworkSectionName}{#1} % Assign \homeworkSectionName to the name of the section from the environment argument
	\subsection{\homeworkSectionName} % Make a subsection with the custom name of the subsection
	\enterProblemHeader{\homeworkProblemName\ [\homeworkSectionName]} % Header and footer within the environment
}{
	\enterProblemHeader{\homeworkProblemName} % Header and footer after the environment
}


%=================================================================

%------------------------------------------------------------------------------------
%	NAME AND CLASS SECTION
%------------------------------------------------------------------------------------

\newcommand{\hmwkTitle}{Assignment\ \#2} % Assignment title
\newcommand{\hmwkClass}{CSC 411} % Course/class
\newcommand{\hmwkAuthorName}{Xiangyu Kong} % Your name
\newcommand{\hmwkUTorId}{kongxi16} % UTorID

%------------------------------------------------------------------------------------
%	TITLE PAGE
%------------------------------------------------------------------------------------

\title{
	\vspace{2in}
	\textmd{\textbf{\hmwkClass:\ \hmwkTitle}}\\
	%	\normalsize\vspace{0.1in}\small{Due\ on\ \hmwkDueDate}\\
	\vspace{0.1in}
	\vspace{3in}
}

\author{\textbf{\hmwkAuthorName} \\ \textbf{\hmwkUTorId}}

% Insert date here if you want it to appear below your name
\date{\today}

%------------------------------------------------------------------------------------

\begin{document}

	\maketitle
	\clearpage

	%---------------------------------------------------------------------------------
	%	PROBLEM 1
	%---------------------------------------------------------------------------------
	\begin{homeworkProblem}
	The data set contains hand-written digits from $0$ to $9$ (Fig \ref{fig:sample}). Among these data, most data are labeled accurately. However, some data are hard to be distinguished and even human can't really predict the digit.(Fig \ref{fig:Incorrect})
	
	\begin{figure}[!ht]
		\centering
		\includegraphics[width=.8\linewidth]{images/1/sample.png}
		\caption{Full data}
		\label{fig:sample}
	\end{figure}

	\begin{figure*}[!ht]
		\begin{subfigure}{.33\textwidth}
			\centering
			\includegraphics[width=.35\linewidth]{images/1/5_0.png}
			\caption{$5$ but looks like a $3$}
			\label{fig:incorrect1}
		\end{subfigure}
		\begin{subfigure}{.33\textwidth}
			\centering
			\includegraphics[width=.35\linewidth]{images/1/6_4.png}
			\caption{$6$ but looks like a $4$}
			\label{fig:incorrect2}
		\end{subfigure}
		\begin{subfigure}{.33\textwidth}
			\centering
			\includegraphics[width=.35\linewidth]{images/1/9_9.png}
			\caption{$9$ but looks like an $8$}
			\label{fig:Incorrect3}
		\end{subfigure}
		\caption{Inaccurate Labels}
		\label{fig:Incorrect}
	\end{figure*}

	\end{homeworkProblem}
	\clearpage

	%---------------------------------------------------------------------------------
	%	PROBLEM 2
	%---------------------------------------------------------------------------------
	\begin{homeworkProblem}
		The output should be:
		\begin{align*}
		o^{(i)} = \sum \limits_{j} w_j x^{(i)}_{j} + b^{(i)}
		\end{align*}
		The listing of the implementation is as follows:\\
		\begin{lstlisting}[language=python, caption=code for linear net output]
	def linear_forward(x, W):
		lin_output = np.dot(W.T, x)
		return softmax(lin_output)
		\end{lstlisting}

	\end{homeworkProblem}
	\clearpage

	%----------------------------------------------------------------------------------
	%	PROBLEM 3
	%---------------------------------------------------------------------------------
	\begin{homeworkProblem}
	
	\begin{enumerate}
		\item 
		Let the loss function $C$ be defined as:
		\begin{align*}
			C &= - \sum \limits_{i} y^{(i)} log ( p_{i} )
		\end{align*}
		
		where $p_{i}$ is:
		\begin{align*}
			p_{i} &= \dfrac{e^{o_i}}{\sum \limits_{j} e^{o_j}}
		\end{align*}
		
		and $o_i$ is:
		\begin{align*}
			o_{i} &= \sum \limits_{j} x^{(i)}_{j} w_{ij} + b_i
		\end{align*}
		
		The gradient for the loss function with respect to the weight $w_{ij}$ $\dfrac{\partial C}{\partial w_{ij}}$ is:
		\begin{align*}
			\dfrac{\partial C}{\partial w_{ij}} = \dfrac{\partial C}{\partial o_{i}} \dfrac{\partial o_{i}}{\partial w_{ij}}
		\end{align*}
		
		Note:
		\begin{align*}
		\dfrac{\partial p_{k}}{\partial o_{i}} &= 
		\begin{cases}
		-p_k p_i       & \textrm{ if } k \neq i \\
		p_i (1 - p_i)  & \textrm{ if } k = i
		\end{cases}
		\end{align*}
		
		Then:
		\begin{align*}
			\dfrac{\partial C}{\partial o_{i}} &= - \dfrac{\partial C}{\partial p_{i}} \dfrac{\partial p_{i}}{\partial o_{i}} + \sum \limits_{k \neq i} \dfrac{\partial C}{\partial p_{k}} \dfrac{\partial p_{k}}{\partial o_{i}} \\
			&= - \dfrac{y^{(i)} }{ p_{i} } p_{i} (1 - p_{i}) + \sum \limits_{k \neq i} \dfrac{y^{(k)}}{p_{k}} p_{k}p_i \\
			&= p_{i} y^{(i)} - y^{(i)} + \sum \limits_{k \neq i} y^{(k)} p_{i} \\
			&= \sum \limits_{k} y^{(k)} p_{i} - y^{(i)} \\
			&= p_{i} - y^{(i)}
		\end{align*}
		
		Also, 
		\begin{align*}
		\dfrac{\partial o_{i}}{\partial w_{ij}} &= x_j^{(i)}
		\end{align*}		
		
		Then
		\begin{align*}
		\dfrac{\partial C}{\partial w_{ij}} &= \dfrac{\partial C}{\partial o_{i}} \dfrac{\partial o_{i}}{\partial w_{ij}} \\
		&= x_{j}^{(i)} (p_{i} - y^{(i)})
		\end{align*}
		
		\newpage
		\item
		The vectorized implementation is as follows:
		\begin{lstlisting}[language=Python, caption=Vectorized gradient]
	def linear_forward(x, W):
		lin_output = np.dot(W.T, x)
		return softmax(lin_output)
	
	def loss(x, W, y):
		p = linear_forward(x, W)
		return -np.sum(y * np.log(p)) / x.shape[1]
	
	def dlossdw(x, W, y):
		p = linear_forward(x, W)
		return np.matmul((p - y), x.T).T
		\end{lstlisting}
		
	\end{enumerate}
	
	
	\end{homeworkProblem}
	\clearpage

	%----------------------------------------------------------------------------------
	%	PROBLEM 4
	%---------------------------------------------------------------------------------
	\begin{homeworkProblem}
		%		\noindent \textit{Question}


	\end{homeworkProblem}
	\clearpage

	%----------------------------------------------------------------------------------
	%	PROBLEM 5
	%---------------------------------------------------------------------------------
	\begin{homeworkProblem}
		%		\noindent \textit{Question}


	\end{homeworkProblem}
	\clearpage

	%----------------------------------------------------------------------------------
	%	PROBLEM 6
	%---------------------------------------------------------------------------------
	\begin{homeworkProblem}
		%		\noindent \textit{Question}


	\end{homeworkProblem}
	\clearpage

	%----------------------------------------------------------------------------------
	%	PROBLEM 7
	%---------------------------------------------------------------------------------
	\begin{homeworkProblem}
		%		\noindent \textit{Question}


	\end{homeworkProblem}
	\clearpage

	%----------------------------------------------------------------------------------
	%	PROBLEM 8
	%---------------------------------------------------------------------------------
	\begin{homeworkProblem}
		%		\noindent \textit{Question}


	\end{homeworkProblem}
	\clearpage

	%----------------------------------------------------------------------------------
	%	PROBLEM 9
	%---------------------------------------------------------------------------------
	\begin{homeworkProblem}
		%		\noindent \textit{Question}


	\end{homeworkProblem}
	\clearpage

	%----------------------------------------------------------------------------------
	%	PROBLEM 10
	%---------------------------------------------------------------------------------
	\begin{homeworkProblem}
		%		\noindent \textit{Question}


	\end{homeworkProblem}
	\clearpage

	%----------------------------------------------------------------------------------

\end{document}
